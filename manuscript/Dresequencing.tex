%% BioMed_Central_Tex_Template_v1.06
%%                                      %
%  bmc_article.tex            ver: 1.06 %
%                                       %

%%IMPORTANT: do not delete the first line of this template
%%It must be present to enable the BMC Submission system to
%%recognise this template!!

%%%%%%%%%%%%%%%%%%%%%%%%%%%%%%%%%%%%%%%%%
%%                                     %%
%%  LaTeX template for BioMed Central  %%
%%     journal article submissions     %%
%%                                     %%
%%          <8 June 2012>              %%
%%                                     %%
%%                                     %%
%%%%%%%%%%%%%%%%%%%%%%%%%%%%%%%%%%%%%%%%%


%%%%%%%%%%%%%%%%%%%%%%%%%%%%%%%%%%%%%%%%%%%%%%%%%%%%%%%%%%%%%%%%%%%%%
%%                                                                 %%
%% For instructions on how to fill out this Tex template           %%
%% document please refer to Readme.html and the instructions for   %%
%% authors page on the biomed central website                      %%
%% http://www.biomedcentral.com/info/authors/                      %%
%%                                                                 %%
%% Please do not use \input{...} to include other tex files.       %%
%% Submit your LaTeX manuscript as one .tex document.              %%
%%                                                                 %%
%% All additional figures and files should be attached             %%
%% separately and not embedded in the \TeX\ document itself.       %%
%%                                                                 %%
%% BioMed Central currently use the MikTex distribution of         %%
%% TeX for Windows) of TeX and LaTeX.  This is available from      %%
%% http://www.miktex.org                                           %%
%%                                                                 %%
%%%%%%%%%%%%%%%%%%%%%%%%%%%%%%%%%%%%%%%%%%%%%%%%%%%%%%%%%%%%%%%%%%%%%

%%% additional documentclass options:
%  [doublespacing]
%  [linenumbers]   - put the line numbers on margins

%%% loading packages, author definitions

%\documentclass[twocolumn]{bmcart}% uncomment this for twocolumn layout and comment line below
\documentclass{bmcart}

%%% Load packages
%\usepackage{amsthm,amsmath}
%\RequirePackage{natbib}
%\RequirePackage[authoryear]{natbib}% uncomment this for author-year bibliography
%\RequirePackage{hyperref}
\usepackage[utf8]{inputenc} %unicode support
%\usepackage[applemac]{inputenc} %applemac support if unicode package fails
%\usepackage[latin1]{inputenc} %UNIX support if unicode package fails

% Modifications for comments
% \usepackage[paperwidth=275.9mm, paperheight=279.4mm]{geometry} %regular letter size is 215.9 wide by 279.44 long
% \setlength{\evensidemargin}{95mm}
%\usepackage[dvipsnames,svgnames,x11names]{xcolor}
\usepackage[markup=underlined]{changes}

\definechangesauthor[color=red]{Corrinne}
\definechangesauthor[color=blue]{Justin}
%%% Alternative definition to have the remarks
%%% in the margins instead of footnotes
\usepackage{todonotes}
\setlength{\marginparwidth}{3cm}
\makeatletter
\setremarkmarkup{\todo[color=Changes@Color#1!20,size=\scriptsize]{#1: #2}}
\makeatother

%% Rather hacky definition of a plain remark/note
%% by riding on \added
\newcommand{\note}[2][]{\added[id=#1,remark={#2}]{}}


%%%%%%%%%%%%%%%%%%%%%%%%%%%%%%%%%%%%%%%%%%%%%%%%%
%%                                             %%
%%  If you wish to display your graphics for   %%
%%  your own use using includegraphic or       %%
%%  includegraphics, then comment out the      %%
%%  following two lines of code.               %%
%%  NB: These line *must* be included when     %%
%%  submitting to BMC.                         %%
%%  All figure files must be submitted as      %%
%%  separate graphics through the BMC          %%
%%  submission process, not included in the    %%
%%  submitted article.                         %%
%%                                             %%
%%%%%%%%%%%%%%%%%%%%%%%%%%%%%%%%%%%%%%%%%%%%%%%%%


\def\includegraphic{}
\def\includegraphics{}



%%% Put your definitions there:
\startlocaldefs
\endlocaldefs


%%% Begin ...
\begin{document}

%%% Start of article front matter
\begin{frontmatter}

\begin{fmbox}
\dochead{Research}

%%%%%%%%%%%%%%%%%%%%%%%%%%%%%%%%%%%%%%%%%%%%%%
%%                                          %%
%% Enter the title of your article here     %%
%%                                          %%
%%%%%%%%%%%%%%%%%%%%%%%%%%%%%%%%%%%%%%%%%%%%%%

%\title{Thirteen genome sequences representing the entire subgenus \textit{Houzingenia} (\textit{Gossypium}): insights into evolution of the New World diploid cottons}
\title{Thirteen genomes}
%%%%%%%%%%%%%%%%%%%%%%%%%%%%%%%%%%%%%%%%%%%%%%
%%                                          %%
%% Enter the authors here                   %%
%%                                          %%
%% Specify information, if available,       %%
%% in the form:                             %%
%%   <key>={<id1>,<id2>}                    %%
%%   <key>=                                 %%
%% Comment or delete the keys which are     %%
%% not used. Repeat \author command as much %%
%% as required.                             %%
%%                                          %%
%%%%%%%%%%%%%%%%%%%%%%%%%%%%%%%%%%%%%%%%%%%%%%

\author[
   addressref={aff1},                   % id's of addresses, e.g. {aff1,aff2}
   corref={aff1},                       % id of corresponding address, if any
   noteref={n1},                        % id's of article notes, if any
   email={corrinne@iastate.edu}   % email address
]{\inits{CE}\fnm{Corrinne E} \snm{Grover}}
\author[
   addressref={aff2},
   email={maa146@igbb.msstate.edu}
]{\inits{MA}\fnm{Mark A} \snm{Arick II}}
\author[
addressref={aff1},
email={jconover@iastate.edu}
]{\inits{JC}\fnm{Justin C} \snm{Conover}}
\author[
addressref={aff3},
email={Jodi.Scheffler@ARS.USDA.GOV}
]{\inits{JA}\fnm{Jodi A} \snm{Scheffler}}
\author[
addressref={aff2},
email={william.s.sanders@gmail.com}
]{\inits{WS}\fnm{William S} \snm{Sanders}}
\author[
addressref={aff2},
email={peterson@IGBB.msstate.edu}
]{\inits{DG}\fnm{Daniel G} \snm{Peterson}}
\author[
addressref={aff3},
email={Brian.Scheffler@ARS.USDA.GOV}
]{\inits{BE}\fnm{Brian E} \snm{Scheffler}}
\author[
addressref={aff1},
email={jfw@iastate.edu}
]{\inits{JF}\fnm{Jonathan F} \snm{Wendel}}

%%%%%%%%%%%%%%%%%%%%%%%%%%%%%%%%%%%%%%%%%%%%%%
%%                                          %%
%% Enter the authors' addresses here        %%
%%                                          %%
%% Repeat \address commands as much as      %%
%% required.                                %%
%%                                          %%
%%%%%%%%%%%%%%%%%%%%%%%%%%%%%%%%%%%%%%%%%%%%%%

\address[id=aff1]{%                           % unique id
  \orgname{Department of Ecology, Evolution, and Organismal Biology, Iowa State University}, % university, etc
  %\street{2200 Osborn Dr},                     %
  %\postcode{50011}                                % post or zip code
  \city{Ames, IA 50011},                              % city
  \cny{USA}                                    % country
}
\address[id=aff2]{%
  \orgname{Institute for Genomics, Biocomputing, and Biotechnology, Mississippi State University},
  %\street{Research Blvd},
  %\postcode{39762}
  \city{Mississippie State, MS 39762},
  \cny{USA}
}
\address[id=aff3]{%
	\orgname{Jamie Whitten Delta States Research Center, USDA-ARS},
	%\street{Research Blvd},
	%\postcode{38776}
	\city{Stoneville, MS 38776},
	\cny{USA}
}


%%%%%%%%%%%%%%%%%%%%%%%%%%%%%%%%%%%%%%%%%%%%%%
%%                                          %%
%% Enter short notes here                   %%
%%                                          %%
%% Short notes will be after addresses      %%
%% on first page.                           %%
%%                                          %%
%%%%%%%%%%%%%%%%%%%%%%%%%%%%%%%%%%%%%%%%%%%%%%

\begin{artnotes}
%\note{Sample of title note}     % note to the article
\note[id=n1]{Equal contributor} % note, connected to author
\end{artnotes}

\end{fmbox}% comment this for two column layout

%%%%%%%%%%%%%%%%%%%%%%%%%%%%%%%%%%%%%%%%%%%%%%
%%                                          %%
%% The Abstract begins here                 %%
%%                                          %%
%% Please refer to the Instructions for     %%
%% authors on http://www.biomedcentral.com  %%
%% and include the section headings         %%
%% accordingly for your article type.       %%
%%                                          %%
%%%%%%%%%%%%%%%%%%%%%%%%%%%%%%%%%%%%%%%%%%%%%%

\begin{abstractbox}

\begin{abstract} % abstract
\parttitle{Background} %if any
Text for this section.

\parttitle{Results} %if any
Text for this section.

\parttitle{Conclusions} %if any
Text for this section.
\end{abstract}

%%%%%%%%%%%%%%%%%%%%%%%%%%%%%%%%%%%%%%%%%%%%%%
%%                                          %%
%% The keywords begin here                  %%
%%                                          %%
%% Put each keyword in separate \kwd{}.     %%
%%                                          %%
%%%%%%%%%%%%%%%%%%%%%%%%%%%%%%%%%%%%%%%%%%%%%%

\begin{keyword}
\kwd{genome sequence}
\kwd{cotton}
\kwd{\textit{Gossypium}}
\kwd{molecular evolution}
\end{keyword}

% MSC classifications codes, if any
%\begin{keyword}[class=AMS]
%\kwd[Primary ]{}
%\kwd{}
%\kwd[; secondary ]{}
%\end{keyword}

\end{abstractbox}
%
%\end{fmbox}% uncomment this for twcolumn layout

\end{frontmatter}

%%%%%%%%%%%%%%%%%%%%%%%%%%%%%%%%%%%%%%%%%%%%%%
%%                                          %%
%% The Main Body begins here                %%
%%                                          %%
%% Please refer to the instructions for     %%
%% authors on:                              %%
%% http://www.biomedcentral.com/info/authors%%
%% and include the section headings         %%
%% accordingly for your article type.       %%
%%                                          %%
%% See the Results and Discussion section   %%
%% for details on how to create sub-sections%%
%%                                          %%
%% use \cite{...} to cite references        %%
%%  \cite{koon} and                         %%
%%  \cite{oreg,khar,zvai,xjon,schn,pond}    %%
%%  \nocite{smith,marg,hunn,advi,koha,mouse}%%
%%                                          %%
%%%%%%%%%%%%%%%%%%%%%%%%%%%%%%%%%%%%%%%%%%%%%%

%%%%%%%%%%%%%%%%%%%%%%%%% start of article main body
% <put your article body there>

%%%%%%%%%%%%%%%%
%% Background %%
%%
\section*{Background}
The American diploid “D-genome” cottons (subgenus \textit{Houzingenia}) comprise a monophyletic clade of cytogenetically and morphologically distinct species largely distributed from Southwest Mexico to Arizona, with additional disjunct species distributions in Peru and the Galapagos Islands \note[Corrinne]{which citations here, Endrizzi et al 1985}. Among the 13 species currently included in the D-genome /note[Corrinne]{although see Ulloa 2013} are \textit{G. raimondii} (D5), the model diploid progenitor to wild and domesticated allopolyploid cotton, and \textit{G. harknessii} (D2-2), an important species for cytoplasmic male sterility in cotton.  Classic cytogenetic research regarding interspecific meiotic pairing patterns revealed that the origin of one allopolyploid progenitor was derived from subgenus \textit{Houzingenia} \note[Corrinne]{Endrizzi 1985, others?}. The close relationship of \textit{Houzingenia} to the agronomically important polyploid species, along with the relative ease of sampling this subgenus for early cotton taxonomists, led to the much of the current understanding of the relationships among D-genome species. 

These early taxonomists divided subgenus \textit{Houzingenia} into two sections and six subsections, whose species alliances have largely been retained by subsequent phylogenetic studies \note[Corrinne]{Cronn et al., 1996; Seelanan et al., 1997; Small and Wendel, 2000b; Wendel and Albert, 1992; Wendel et al., 1995b; alvarez 2005}. Several molecular datasets have been used to evaluate these relationships, including chloroplast restriction sites \note[Corrinne]{citation}; simple sequence repeat (SSR) and expressed sequence tag (EST)-SSR markers \note[Corrinne]{citation}; random amplified polymorphic DNA (RAPD) markers \note[Corrinne]{citation}; internal transcribed sequences (ITS) \note[Corrinne]{citation}; and few single-copy nuclear genes \note[Corrinne]{citation}. Relationships among the six subsections, however, remain unclear despite numerous, and often conflicting, studies \note[Corrinne]{Cronn et al., 1996; Liu et al., 2001b; Small and Wendel, 2000b more citations}. Determining the closest living relative of the D-genome ancestor to the polyploid, however, has been met with greater success. Early morphological and cytogenetic comparisons using intergenomic hybrids quickly identified G. raimondii as the closest living relative to the D-genome ancestor of polyploid cotton species (Stephens, 1944b Hutchinson et al., 1945, 1947 ; Gerstel and Phillips, 1958; Phillips, 1964; reviewed in Wendel and Cronn 2003). Subsequent analyses have largely supported this observation (Abdalla et al., 2001; cronn 1999, liu 2001, Cronn et al., 1996 Seelanan et al., 1997 Small et al., 1998; Small and Wendel, 2000a,b), with few conflicts (see, however Wendel et al., 1995b; I know there are others…. What are they?).

%\cite{koon,oreg,khar,zvai,xjon,schn,pond,smith,marg,hunn,advi,koha,mouse}

\section*{Results}
Text for this section \ldots
\subsection*{Results subsection}
Text for this sub-heading \ldots

\section*{Discussion}

\section*{Conclusions}

\section*{Methods}

%\nocite{oreg,schn,pond,smith,marg,hunn,advi,koha,mouse}

%%%%%%%%%%%%%%%%%%%%%%%%%%%%%%%%%%%%%%%%%%%%%%
%%                                          %%
%% Backmatter begins here                   %%
%%                                          %%
%%%%%%%%%%%%%%%%%%%%%%%%%%%%%%%%%%%%%%%%%%%%%%

\begin{backmatter}

\section*{Competing interests}
  The authors declare that they have no competing interests.

\section*{Author's contributions}
    Text for this section \ldots

\section*{Acknowledgements}
  We would like to thank....
  
  
%%%%%%%%%%%%%%%%%%%%%%%%%%%%%%%%%%%%%%%%%%%%%%%%%%%%%%%%%%%%%
%%                  The Bibliography                       %%
%%                                                         %%
%%  Bmc_mathpys.bst  will be used to                       %%
%%  create a .BBL file for submission.                     %%
%%  After submission of the .TEX file,                     %%
%%  you will be prompted to submit your .BBL file.         %%
%%                                                         %%
%%                                                         %%
%%  Note that the displayed Bibliography will not          %%
%%  necessarily be rendered by Latex exactly as specified  %%
%%  in the online Instructions for Authors.                %%
%%                                                         %%
%%%%%%%%%%%%%%%%%%%%%%%%%%%%%%%%%%%%%%%%%%%%%%%%%%%%%%%%%%%%%

% if your bibliography is in bibtex format, use those commands:
\bibliographystyle{bmc-mathphys} % Style BST file (bmc-mathphys, vancouver, spbasic).
\bibliography{bmc_article}      % Bibliography file (usually '*.bib' )
% for author-year bibliography (bmc-mathphys or spbasic)
% a) write to bib file (bmc-mathphys only)
% @settings{label, options="nameyear"}
% b) uncomment next line
%\nocite{label}

% or include bibliography directly:
% \begin{thebibliography}
% \bibitem{b1}
% \end{thebibliography}

%%%%%%%%%%%%%%%%%%%%%%%%%%%%%%%%%%%
%%                               %%
%% Figures                       %%
%%                               %%
%% NB: this is for captions and  %%
%% Titles. All graphics must be  %%
%% submitted separately and NOT  %%
%% included in the Tex document  %%
%%                               %%
%%%%%%%%%%%%%%%%%%%%%%%%%%%%%%%%%%%

%%
%% Do not use \listoffigures as most will included as separate files

\section*{Figures}
  \begin{figure}[h!]
  \caption{\csentence{Sample figure title.}
      A short description of the figure content
      should go here.}
      \end{figure}

\begin{figure}[h!]
  \caption{\csentence{Sample figure title.}
      Figure legend text.}
      \end{figure}

%%%%%%%%%%%%%%%%%%%%%%%%%%%%%%%%%%%
%%                               %%
%% Tables                        %%
%%                               %%
%%%%%%%%%%%%%%%%%%%%%%%%%%%%%%%%%%%

%% Use of \listoftables is discouraged.
%%
\section*{Tables}
\begin{table}[h!]
\caption{Sample table title. This is where the description of the table should go.}
      \begin{tabular}{cccc}
        \hline
           & B1  &B2   & B3\\ \hline
        A1 & 0.1 & 0.2 & 0.3\\
        A2 & ... & ..  & .\\
        A3 & ..  & .   & .\\ \hline
      \end{tabular}
\end{table}

%%%%%%%%%%%%%%%%%%%%%%%%%%%%%%%%%%%
%%                               %%
%% Additional Files              %%
%%                               %%
%%%%%%%%%%%%%%%%%%%%%%%%%%%%%%%%%%%

\section*{Additional Files}
  \subsection*{Additional file 1 --- Sample additional file title}
    Additional file descriptions text (including details of how to
    view the file, if it is in a non-standard format or the file extension).  This might
    refer to a multi-page table or a figure.

  \subsection*{Additional file 2 --- Sample additional file title}
    Additional file descriptions text.


\end{backmatter}
\end{document}
